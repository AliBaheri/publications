% !TEX root = uai2015.tex

\section{CONCLUSION}
\label{sec:conclusion}

Mismatched models exhibit poor predictive performance. The \gls{POPEB} predictive
density mitigates this effect by incorporating the population $F$ into
Bayesian analysis. The \gls{POPEB} \gls{MAP} predictive density is attractively
simple; it explores the space of bootstrapped dataset to find the one with
highest predictive power. \gls{BUMPVI} extends this idea to variational
inference; it delivers an efficient algorithm with promising results on
real-world datasets.

\gls{POPEB}, like \gls{EB}, uses the dataset twice:
once to estimate the prior and again during inference. There are alternatives to
consider, such as cross-validation and bias correction
\citep{Efron:1997,gelman2013understanding}.
Directly modeling the expected predictive performance of future data should also
improve \gls{POPEB}.

\begin{table}[!htb]
\caption{Coordinate ascent \gls{LDA} topics ($K=8$).}
\label{tab:coordinate_ascent}
\begin{center}
{\fontsize{8pt}{8pt}\selectfont
\input{tab/lda_coord_ascent.tex}
}
\end{center}
\end{table}

\begin{table}[!hbt]
\caption{\gls{BUMPVI} \gls{LDA} topics ($K=8$).}
\label{tab:bumping}
\begin{center}
{\fontsize{8pt}{8pt}\selectfont
\input{tab/lda_bumping.tex}
}
\end{center}
\end{table}
