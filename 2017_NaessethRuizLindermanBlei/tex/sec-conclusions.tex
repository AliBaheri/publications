%!TEX root = ./main.tex

\section{Conclusions}
\glsresetall

We introduced \gls{RS-VI}, a method for deriving reparameterization
gradients when simulation from the variational distribution is done
using a acceptance-rejection sampler.  In practice, \gls{RS-VI} leads to
lower-variance gradients than other state-of-the-art methods.
Further, it enables reparameterization gradients for a large class of
variational distributions, taking advantage of the efficient
transformations developed in the rejection sampling literature.

This work opens the door to other strategies that ``remove the lid''
from existing black-box samplers in the service of variational
inference.  As future work, we can consider more complicated
simulation algorithms with accept-reject-like steps, such as adaptive
rejection sampling, importance sampling, sequential Monte Carlo, or
Markov chain Monte Carlo.

% ? This is an intriguing question which we will have to leave for
% future work.

% Extendend space to do inference for more expressive models:
% \citet{Salimans2015,TranNK2015,Tran2016,Maaloe2016,Ranganath2016,Moreno2016}?

% Perhaps we could mention learning a mapping $h(\eps,\theta)$ online
% like in normalizing flows as future work? Could we be able to apply
% this more generally to perhaps MCMC methods with accept--reject
% steps?

%%% Local Variables:
%%% mode: latex
%%% TeX-master: "main"
%%% End:
